\section{Diskussion}
Bei der Fourier-Analyse fällt auf, dass bei der Rechteckspannung hohe Abweichungen
vorliegen. Möglicher Grund ist, dass bei der Messung
ein Rauschen im Hintergrund zu sehen war und somit das Oszilloskop die Signale
nicht richtig umwandeln konnte. Da die beiden anderen Spannungsformen im  Toleranzbereich
liegen könnte bei der Rechteckspannung ein systematischer Fehler vorliegen.
Bei der Fourier-Synthese sind gute Ergebnisse erzielt worden, was man an anhand der Bilder
erkennen kann. Bei der Dreieckspannung ist die Synthese am besten gelungen, da die
Amplitude mit den Faktor $\frac{1}{n^2}$ abfällt. 
