\section{Theorie}

In der Physik sind periodische Vorgänge sehr wichtig. Es lassen sich fast alle
periodische Vorgänge, die in der Natur vorkommen, beschreiben durch das Fouriersche
Theorem. Dieses Theorem besagt, dass die in Gleichung (\ref{eq:1}) gezeigte Reihe eine
beliebige periodische Funktion darstellt, falls sie gleichmäßig konvergiert.

\begin{equation}
  \frac{1}{2} a_0 + \sum^{\infty}_{n=1} \left( a_n \cos(\frac{2\pi n}{T} t) + b_n \sin(\frac{2\pi n}{T}t) \right)
  \label{eq:1}
\end{equation}

Die Koeffizienten $a_n$ und $b_n$ können mit der folgenden Formel bestimmt werden:

\begin{align}
  a_n = \frac{2}{T} \int_0^T f(t) \cos(\frac{2\pi n}{T} t) \, \symup{d}t &&
  b_n = \frac{2}{T} \int_0^T f(t) \sin(\frac{2\pi n}{T} t) \, \symup{d}t
  \label{eq:2}
\end{align}

Bei der Bestimmung der Amplituden ist es wichtig ob die zu beschreibende Funktion gerade oder
ungerade ist. Ist die Funktion gerade, also ist $f(t)=f(-t)$, dann sind alle $b_n=0$.
Falls die Funktion ungerade ist, also $f(t)=-f(-t)$ gilt, sind alle $a_n=0$. 
