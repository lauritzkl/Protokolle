\section{Diskussion}
Dadruch das die eingestellte Phase nicht geeicht war, sind die meisten Messungen deren Fehler
im Toleranzbereich. Die dennoch vorhandene Abweichungen können durch ein systematischen Fehler
argumetiert werden. Die Abweichungen der Tabelle (\ref{tab:2}) sind auch durch den Noise Generator
zu erklären.
Mit Hilfe der Ausgleichsrechnung ist die tatsächliche Phase um ca. $11 \, \text{Grad}$ verschoben.
Die Amplitude, die $\frac{U_0}{2} = 55,6 \, V$ ist, ergibt sich aus der Ausgleichsrechnung eine neue Amplitude von $58,3 \, V$.
Dies ist eine Abweichung von ca. $4,87 \, \%$. \\
Zur Überprüfung des Strahlungsintensitätsgesetzes mithilfe einer Leuchtdiode sind die Werte bis zu einem
Abstand von $r = 14,5 \, cm$ nicht physikalisch sinnvoll, da sie quadratisch abfallen müssten.
Mögliche Gründe für solche Werte sind, dass Störfaktoren wie Deckenlampen die
Messwerte verfälschen.
