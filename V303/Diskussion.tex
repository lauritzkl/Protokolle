\section{Diskussion}
Die Abweichungen der Tabelle (\ref{tab:1}) sind im ersten Theorieteil sehr hoch.
Dies führt dazu, dass die eingestellte Phasendifferenz nicht geeicht war.
Mit Hilfe der Ausgleichsrechnung ist die tatsächliche Phase um ca. $11 \text{Grad}$ verschoben.
Somit sind die Abweichungen deutlich geringer gegenüber zum ersten Theorieteil.
Die dennoch vorhandenen Abweichungen lassen sich aus systematische Fehler rückfolgern.
Die Amplitude, die $55,6 \, V$ ist, ergibt sich aus der Ausgleichsrechnung eine neue Amplitude von $58,3 \, V$.
Dies ist eine Abweichung von ca. $4,87 \, \%$.
Die Abweichungen der Tabelle (\ref{tab:2}) sind durch den Noise Generator zu erklären. Sie führen ebenfalls zu systematische Fehlern.\\
Zur Überprüfung des Strahlungsintensitätsgesetzes mithilfe einer Leuchtdiode sind die Werte bis zu einem
Abstand von $14,5 \, cm$ nicht physikalisch sinnvoll. Mögliche Gründe für solche Werte sind, dass Störfaktoren wie Deckenlampen die
Messwerte verfälschen.
