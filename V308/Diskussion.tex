\section{Diskussion}

Bei dem Vergleich der magnetischen Flussdichte der beiden Spulen fällt auf, dass
die Theoriekurve unterhalb der gemessenen Werten liegt. Das kann daran liegen, dass
der Strom der Spule nicht genau ermittelt wurde durch Parallaxefehler. Außerdem
musste bei der langen Spule von beiden Seiten seperat gemessen werden und die jeweiligen
Messwerte haben sich nie überschnitten, was erklärt warum die Messwerte im inneren
der Spule einen Sprung von \SI{0.1}{\milli\tesla}. Ein weiterer Grund für das Abweichen der
Messwerte ist, dass nicht genau bestimmt werden konnte ab wann in der Spule gemessen wurde.

Die ersten Graphen des Spulenpaares geht hervor, dass bei dem Abstand \SI{0.07}{\meter}
das Magnetfeld innerhalb der Spule sehr homogen ist, was die Theorie auch ergeben hat.
Da der Abstand nicht genau der Radius der Spulen ist, ergibt sich eine leichte Abweichung
von den Theoriewerten. Auch bei den breiteren Abständen wird gut sichtbar, dass
das Magnetfeld zwischen den Spulen immer inhomogener wird, wie es in der Theorie
erwartet wird. Das wird auch gut sichtbar anhand der prozentualen Abweichungen, die
$2,97 \% , 0,28\% , 6,04 \%$ betragen.

Wie vorher erwähnt ergibt sich aus der Hysteresekurve die Sättigungsmagnetisierung
zu $B_s = \SI{694.2}{\milli\tesla}$, die Remanenz zu $B_r = \SI{91.5(350)}{\milli\tesla}$ und
die Koerzitivkraft zu $I_c = \SI{-0.6}{\ampere}$.
Bei der Hysteresekurve fällt auf, dass es bei der Remanenz einen Sprung des Magnetfeldes
gibt, weil der Generator nur eine Spannung von $0-10\si{\volt}$ erzeugen kann und
er bei der Remanenz umgepolt werden muss. Außerdem konnte das Material nicht vollständig
entmagnetisiert werden, was auch zu Fehlern geführt haben könnte.
