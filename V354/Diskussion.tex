\section{Diskussion}

Bei der Bestimmung des Dämpfungswiderstandes $R_{eff}$ fällt auf, dass der gemessene
Widerstand um $\SI{50.5}{\ohm}$ von den eingebauten Widerstand abweicht. Das liegt
daran, dass der Innenwiderstand des Generators nicht berücksichtigt wurde.

Die Abweichung des Widerstandes bei dem aperiodischen Grenzfall ist sehr groß.
Das kann zum einen daran liegen, dass das Oszilloskop nicht genau genug gemessen hat um
den Widerstand genau zu bestimmen. Zum anderen wurde bei der theoretischen Berechnung
der gesamte Widerstand der Schaltung bestimmt wurde und die anderen Bauelement auch
Widerstände haben wurde nicht betrachtet.

Die restlichen Abweichungen sind im Toleranzbereich. Bei der Bestimmung der Resonanzüberhöhung
ist die Abweichung etwas höher, was daran liegen kann, dass das Maximum der Spannung am
Kondensator nicht genau bestimmt wurde.
