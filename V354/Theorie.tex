\section{Zielsetzung}
In diesem Versuch soll der effektive Dämpfungswiderstand
einer gedämpften Schwingkreisschaltung als auch beim aperiodischen Grenzfall
untersucht und ermittelt werden. Ebenso wird die Frequenzabhängigkeit
der Kondensatorspannung und die Phasenverschiebung von der außen gelegten Spannung
untersucht.
\section{Theorie}
Ein $CL$-Schwingkreis, das aus der Kapazität $C$ des Kondensators und der
Induktivität $L$ der Spule besteht, wird in der Physik durch den
Energieaustausch der beiden Baudelemente als eine periodische Schwingung bezeichnet.
Solange kein energieverbrauchendes Bauelement hinzukommt ist der Austausch unbegrenzt lang
und wird als ungedämpfte Schwingung bezeichnet.
Bei einer gedämpften Schwingung wird als energieverbrauchendes Bauelement ein ohmscher Widerstand $R$
hinzugeschaltet wie in Abbildung (\ref{fig:1}) zu sehen ist.
\begin{figure}[H]
\centering
\includegraphics[width=10 cm , height=7 cm]{Schwingkreis.png}
\caption{Schaltdarstellung einer gedämpften Schwigung[2].}
\label{fig:1}
\end{figure}
Nun beginnt kein unendlicher Energieaustausch zwischen der Kapazität $C$
des Kondensators und der Induktivität $L$ der Spule statt.
Die Spannungen die an den einzelnen Bauelementen abfallen,
können mit Hilfe der Maschenregel, zu einer Differentialgleichung aufgestellt und umgeformt werden.
\begin{equation}
  \ddot{I} + \frac{R}{L} \cdot \dot{I} +\frac{1}{LC}\cdot I = 0
  \label{eq:1}
\end{equation}
Die Lösung der Gleichung (\ref{eq:1}) lautet:
\begin{equation}
  I(t) = A_1 \cdot e^{i\omega_1 t} + A_2 \cdot e^{i\omega_2 t}
  \label{eq:2}
\end{equation}
Dabei ist
\begin{equation*}
  \omega_{1,2} =  i \frac{R}{2L} \pm \sqrt{\frac{1}{LC} - \frac{R^2}{4L^2}}
\end{equation*}
Es werden nun 2 Fälle betrachtet.\\

Der erste Fall $\frac{1}{LC} \gg \frac{R^2}{4L^2}$ folgt, dass $\omega$ reell ist.\\
Somit wird die Gleichung(\ref{eq:2}) mit Hilfe der Eulerschen Form zu
\begin{equation}
  I(t) = A_0 \cdot e^{-2\pi\mu t} \cdot cos(2\pi\nu t + \varphi)
  \label{eq:3}
\end{equation}
umgeschrieben. Dabei ist
\begin{align*}
    \mu &= \frac{R}{4\pi L}& \\\\
    \nu &= \frac{1}{2\pi} \sqrt{\frac{1}{LC} - \frac{R^2}{4L^2}}&
\end{align*}
Die Gleichung (\ref{eq:3}) stellt eine gedämpfte Schwingung da. Die Amplitude nimmt nach zunehmnder Zeit exponentiell ab.
und somit kann man für die Schwingunsdauer folgende Formel verwenden:
\begin{equation}
  T_ex = \frac{1}{2\pi \mu} = \frac{2L}{R}
  \label{eq:4}
\end{equation}
Der zweite Fall $\frac{1}{LC} \ll \frac{R^2}{4L^2}$ folgt daraus, dass $\omega$ imaginär ist
und somit sich
\begin{equation*}
I(t) \sim e^{-(2\pi \mu - i 2 \pi \nu)t}
\end{equation*} verhält.\\
Ein Spezial Fall ist, wenn
\begin{equation}
\frac{1}{LC} = \frac{R_{ap}^2}{4L^2}
\label{eq:5}
\end{equation}
folgt daraus das der aperiodische Grenzfall eintrifft und der Strom am schnellsten gegen null geht.
\newpage
Erzwungene Schwingungen werden mit einer Sinusspannung, wie in Abbildung(\ref{fig:2}) dargestellt, erzeugt.
\begin{figure}[H]
\centering
\includegraphics[width=10 cm, height= 7 cm]{Schwingkreis2.png}
\caption{Schaltdarstellung einer erzwungene Schwinung[2].}
\label{fig:2}
\end{figure}

Dabei verändert sich die Gleichung (\ref{eq:1}) und die Lösung für solch eine Differentialgleichung lautet:
\begin{equation}
  U_c(\omega) = \frac{U_0}{\sqrt{(1-LC\omega^2 + (RC\omega)^2}}
  \label{eq:5}
\end{equation}
Die Phasenverschiebung zwischen der angelegten Sinusspannung und der Kondensatorspannung kann mit der Formel
\begin{equation}
\varphi(\omega) = \arctan(\frac{-\omega CR}{1-LC\omega^2})
\label{eq:4}
\end{equation}
errechnet werden.\\
Für $\omega \rightarrow \infty$ ist $U_c$ = 0 und für $\omega \rightarrow 0$ strebt $U_c$ gegen $U_o$.
Ab einer bestimmten Frequenz erreicht $U_c$
ein Maximum und ist größer als $U_o$.
Solch ein Phänomen bezeichnet man als Resonanz.
Die Resonanzfrequenz lässt sich mit der Formel
\begin{equation}
  \omega_{res} = \sqrt{\frac{1}{LC} - \frac{R^2}{2L^2}}
  \label{eq:7}
\end{equation}
berechnen.
Bei einer schwachen Dämpfung trifft der Fall, dass $\frac{1}{LC} \ll \frac{R^2}{2L^2}$ das $\omega_{res} \rightarrow \omega_0$ wird.
Somit lässt sich die Formel mit Hilfe der Gleichung (\ref{eq:7})
\begin{equation}
  U_{c,max} = \frac{1}{\omega RC} \cdot U_0
\end{equation}
darstellen. Der Faktor $\frac{1}{\omega_0 RC}$ wird als Resonazüberhöhung oder Güte $q$ bezeichnet.
