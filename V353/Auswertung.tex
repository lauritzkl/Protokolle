
\begin{document}
\section{Auswertung}
\subsection{Bestimmung der Zeitkonstante über Auf- und Entladungsvorgang}
Zur Bestimmung der Zeitkonstante $RC$ werden die Messdaten wie in Tabelle(\ref{tab:1})
in ein Diagramm (\ref{fig:1}) dargestellt und mit Hilfe einer linearen Ausgleichsrechnung
berechnet.
\begin{table}[H]
  \centering
  \caption{Tabelle zur Bestimmung der Zeitkonstante mit $U_\text{0}$ = $10V$}
    \begin{tabular}{c c c }
      \toprule \\
      $U_\text{c} / V$& $ln(\frac{U_\text{c}}{U_\text{0}})$ & $t /\si{\milli\second}$ \\
      \midrule \\
      10,0& -0.000 & 0.00 \\
      9,04& -0.100 & 0.10 \\
      8,48& -0.165 & 0.16 \\
      7,84& -0.243 & 0.22 \\
      7,12& -0.340 & 0.34 \\
      6,80& -0.386 & 0.40 \\
      6,16& -0.485 & 0.50 \\
      5,36& -0.624 & 0.66 \\
      4,88& -0.717 & 0.78 \\
      4,32& -0.839 & 0.96 \\
      3,76& -0.978 & 1.14 \\
      3,52& -1.044 & 1.24 \\
      3,28& -1.115 & 1.36 \\
      3,04& -1.191 & 1.50 \\
      2,72& -1.302 & 1.80 \\
      2,48& -1.394 & 1.96 \\
      2,32& -1.461 & 2.18 \\
      2,24& -1.496 & 2.30 \\
      2,16& -1.532 & 2.56 \\
      2,08& -1.570 & 2.80 \\
      2,08& -1.570 & 3.02 \\
      2,00& -1.609 & 3.18 \\
      2,00& -1.609 & 3.36 \\
      2,00& -1.609 & 3.66 \\
      2,00& -1.609 & 3.86 \\
      2,00& -1.609 & 4.10 \\
      2,00& -1.609 & 4.26 \\
      2,00& -1.609 & 4.28 \\
      2,00& -1.609 & 4.32 \\
      \bottomrule
    \end{tabular}
    \label{tab:1}
  \end{table}

\begin{figure}[H]
  \centering
  \includegraphics[width=\textwidth]{Diagramm1.pdf}
  \caption{Diagrammdarstellung}
  \label{fig:1}
\end{figure}
Die Ausgleichsrechnungs allgemein lautet:
\begin{align}
  y & = m \cdot x + b \label{eq:}\\
  m & = \frac {\bar{xy} - \bar{x} \cdot \bar{y}} {\bar{x^2} -\bar{x}^2}&  \label{eq:}\\
  b & = \frac {\bar{y} \cdot \bar{x}^2 - \bar{xy} \cdot \bar{x}} {\bar{x^2}-\bar{x}^2}& \label{eq:}
\end{align}
Für diese Ausgleichsrechnung wird die Formel (\ref{eq:1}) umgeschrieben und die erechnetet Werte lauten: \\
\newline
\centerline{$ln(\frac{U_\text{c}}{U_\text{0}}) = -\frac{1}{m} + b$}\\
\newline
\centerline{mit $m = (2,845 \pm 0,245) \cdot 10^{-3} \si{\second}$}\\
\newline
\centerline{und $b = (-0,393 \pm 0,074)$}
\newline
\subsection{Bestimmung der Zeitkonstante über den Tiefpassvorgang}
Dabei wird die normierte Amplitude in Abhängigkeit von der Frequenz wie in der Tabelle 2 in
einen Diagramm dargestellt.
\begin{table}[H]
  \centering
  \caption{Tabelle von der Amplitude in Abhängigkeit der Frequenz mit $U_0 = 10V$}
    \begin{tabular}{c c}
      \toprule \\
      $\frac{A(\omega)}{U_\text{0}}$ & $\omega \,\, \text{in} \,\, Hz$ \\
      \midrule \\
      0.448 & 10\\
      0.448 & 20\\
      0.448 & 30\\
      0.440 & 40\\
      0.440 & 50\\
      0.432 & 60\\
      0.424 & 70\\
      0.416 & 80\\
      0.416 & 90\\
      0.400 & 100\\
      0.328 & 200\\
      0.264 & 300\\
      0.216 & 400\\
      0.176 & 500\\
      0.152 & 600\\
      0.135 & 700\\
      0.120 & 800\\
      0.112 & 900\\
      0.096 & 1000\\
      0.0448& 2000\\
      0.0296& 3000\\
      0.0224& 4000\\
      0.0180& 5000\\
      0.0152& 6000\\
      0.0128& 7000\\
      0.0112& 8000\\
      0.0100& 9000\\
      0.0090& 10000\\
      \bottomrule
    \end{tabular}
    \label{tab:2}
  \end{table}

\begin{figure}[H]
  \centering
  \includegraphics[width=\textwidth]{Diagramm2.pdf}
  \caption{Diagrammdarstellung}
  \label{fig:2}
\end{figure}
Für die nicht-lineare Ausgleichsrechnung wird die Gleichung (7) verwendet.
\begin{equation*}
  \frac{A}{U_0} = \frac{1}{\sqrt{1 + (\omega m)^2}}
\end{equation*}
\centerline{mit $m = (4,46 \pm \, 0.07) \cdot 10^{-3} \si{\second}$}
\subsection{Bestimmung der Zeitkonstante mit Hilfe der Phasenverschiebung}
Es werden die Daten von der Tabelle 3 in einen Diagramm dargestellt und mit Hilfe
einer nicht-linearen Ausgleichsrechnung die Zeitkonstante bestimmt.
\input{Tab3.tex}
\begin{figure}[H]
  \centering
  \includegraphics[width=\textwidth]{Diagramm3.pdf}
  \caption{Diagrammdarstellung}
  \label{fig:3}
\end{figure}
Dabei wird die Gleichung () verwendet und umgeschrieben:
\begin{equation*}
  \phi= arctan(m*\omega)
\end{equation*}
\centerline{Dabei ist $m=(5,228 \pm \, \inf) \cdot 10^{-3} \si{\second}$}
Nun wird die relative Amplitude in Abhängigkeit der Phase in einen Polarkoordinaten aufgetragen mit
der vorhin ermittelten $RC-Wert = (5,228 \pm \, \inf) \cdot 10^{-3}$.
\begin{figure}[H]
  \centering
  \includegraphics[width=\textwidth]{Polar.pdf}
  \label{fig:4}
\end{figure}
\subsection{Integrator}
In diesem Fall wird die Gleichung () verwendet. \\
\centerline{1.Fall Sinusspannung}
\begin{equation*}
  $f(x) = A * sin(x) \rightarrow F(x)= - A * cos(x)$
\end{equation*}
\begin{figure}[H]
  \centering
  \includegraphics[width=\textwidth]{Sinusspannung.BMP}
  \label{fig:5}
\end{figure}
Man erkennt auf dem Thermodruck das die Integration über die Funktion f(x) die
Stammfunktion F(x) ergibt.
2. Fall Dreieckspannun
\begin{equation*}
  f(x) =
  \begin{cases}
    c \cdot x \. \text{für} \. $-a \leq x \leq a$ \\
  - c \cdot x \. \text{für} \. $a \leq x \leq 3a$
  \end{cases}
\end{equation*}
Die Integrationen für diese Funktion f(x) lautet:
\begin{equation*}
  F(x)
  \begin{cases}
    \frac{c}{2} x^2 \text{für} -a \leg x \leg a \\
    \frac{-c}{2} x^2 \text{für} a \leg x \leg 3a
  \end{cases}
\end{equation*}
\begin{figure}[H]
  \centering
  \includegraphics[width=\textwidth]{Dreieckspannung.BMP}
  \label{fig:6}
\end{figure}
\end{document}
