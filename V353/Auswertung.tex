\input{header.tex}
\begin{document}
\section{Auswertung}
\subsection{Bestimmung der Zeitkonstante über Auf- und Entladungsvorgang}
Zur Bestimmung der Zeitkonstante $RC$ werden die Messdaten wie in (\ref{tab:1})
in ein Diagramm (\ref{fig:1}) dargestellt und mit Hilfe einer linearen Ausgleichsrechnung
berechnet.
\begin{table}[H]
  \centering
  \caption{Tabelle zur Bestimmung der Zeitkonstante mit $U_\text{0}$ = $10V$}
    \begin{tabular}{c c c }
      \toprule \\
      $U_\text{c} / V$& $ln(\frac{U_\text{c}}{U_\text{0}})$ & $t /\si{\milli\second}$ \\
      \midrule \\
      10,0& -0.000 & 0.00 \\
      9,04& -0.100 & 0.10 \\
      8,48& -0.165 & 0.16 \\
      7,84& -0.243 & 0.22 \\
      7,12& -0.340 & 0.34 \\
      6,80& -0.386 & 0.40 \\
      6,16& -0.485 & 0.50 \\
      5,36& -0.624 & 0.66 \\
      4,88& -0.717 & 0.78 \\
      4,32& -0.839 & 0.96 \\
      3,76& -0.978 & 1.14 \\
      3,52& -1.044 & 1.24 \\
      3,28& -1.115 & 1.36 \\
      3,04& -1.191 & 1.50 \\
      2,72& -1.302 & 1.80 \\
      2,48& -1.394 & 1.96 \\
      2,32& -1.461 & 2.18 \\
      2,24& -1.496 & 2.30 \\
      2,16& -1.532 & 2.56 \\
      2,08& -1.570 & 2.80 \\
      2,08& -1.570 & 3.02 \\
      2,00& -1.609 & 3.18 \\
      2,00& -1.609 & 3.36 \\
      2,00& -1.609 & 3.66 \\
      2,00& -1.609 & 3.86 \\
      2,00& -1.609 & 4.10 \\
      2,00& -1.609 & 4.26 \\
      2,00& -1.609 & 4.28 \\
      2,00& -1.609 & 4.32 \\
      \bottomrule
    \end{tabular}
    \label{tab:1}
  \end{table}

\begin{figure}[H]
  \centering
  \includegraphics[width=\textwidth]{Diagramm1.pdf}
  \caption{Diagrammdarstellung}
  \label{fig:1}
\end{figure}
Die Ausgleichsrechnungs allgemein lautet:
\begin{align}
  y & = m \cdot x + b \label{eq:1}\\
  m & = \frac {\bar{xy} - \bar{x} \cdot \bar{y}} {\bar{x^2} -\bar{x}^2}&  \label{eq:2}\\
  b & = \frac {\bar{y} \cdot \bar{x}^2 - \bar{xy} \cdot \bar{x}} {\bar{x^2}-\bar{x}^2}& \label{eq:3}
\end{align}
Für diese Ausgleichsrechnung wird die Formel (\ref{eq:1}) umgeschrieben und die erechnetet Werte lautetn: \\
\newline
\centerline{$ln(\frac{U_\text{c}}{U_\text{0}}) = -\frac{1}{m} + b$}\\
\centerline{mit $m = (2,845 \pm 0,245) \cdot 10^{-3} \si{\second}$}\\
\centerline{und $b = (-0,393 \pm 0,074)$}
\newline
\end{document}
