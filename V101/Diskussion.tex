\section{Diskussion}
Zwischen dem theoretisch und experimentell bestimmten Trägheitsmoment sind die Werte weit auseinander.
Das kann daran liegen, dass die Winkelrichtsgröße D zu Beginn fehlerhaft bestimmt worden ist
und somit sich der Fehler fortsetzt. Außerdem könnte auch das Trägheismoment der Drillachse
zu groß bestimmt worden sein und da man zur Bestimmung der Trägheitsmomente das Trägheitsmoment
der Drillachse abziehen muss kann dies zu den negativen Werten geführt haben.
Bei der Bestimmung des Trägheitsmoments der Drillachse wurde die Metallstange
als Masselos angenommen, was auch eine Fehlerquelle des Trägheitsmomentes ist.
Das Messen der Schwingungsdauer der Objekte war zum Teil sehr schwierig und die
Messuhren haben nicht genau dann ausgelößt, wenn sie betätigt wurden. Dadurch ist
der Fehler der Trägheitsmomente groß geworden und das könnte auch dazu beigetragen
haben, dass die Trägheitsmomente negativ geworden sind.
Es gibt keine negativen Trägheitsmomente, da sie physikalisch nicht sinnvoll sind
bzw. nicht möglich sind.
Die Näherung der Körperteile der Puppe als Zylinder ist ungenau, wodurch der
theoretisch bestimmte Wert nicht sehr genau sein kann.
Die Messungen könnten Verbessert werden, durch geeichte Federwaagen mit denen
genau senkrecht gemessen werden kann.
