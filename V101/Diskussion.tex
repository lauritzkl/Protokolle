\section{Diskussion}
Zwischen dem theoretischen und experimentellen Trägheitsmoment sind die Werte weit auseinander. Das kann dazu führen, dass die Winkelgröße D zu Beginn falsch bestimmt worden ist und somit sich der Fehler fortsetzt.
Ebenso gibt es kein negatives Trägheitsmoment, da sie physikalisch nicht sinnvoll sind bzw. nicht möglich sein können. Der mögliche Fehler liegt an der Bestimmung des Trägheitsmomentes der Drillachse.
Um die experimentelle Messwerte mit dem theoretischen Werte vergleichen zu können müsste die
Winkelgröße D  gemittelt werden um den Fehler zu verringern.
