\section{Diskussion}

Bei der Bestimmung der Leerlaufspannung durch die Ausgleichsrechnung bei der Monozelle
sind die Abweichung zu der gemessenen Leerlaufspannung $\SI{1.21}{\percent}$ und
$\SI{7.27}{\percent}$. Sie liegen also im Toleranzbereich und da der systematische
Fehler der durch den endlichen Widerstand des Voltmeters entsteht vernachlässigbar ist,
wurde die Leerlaufspannung und damit auch der Innenwiderstand gut gemessen.

Auch bei der umgesetzten Leistung am Wirkwiderstand liegen alle Messwerte im Toleranzbereich,
da die größte Abweichung \SI{6.43}{\percent} ist.
